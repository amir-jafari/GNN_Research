\documentclass{beamer}
%\documentclass[17pt]{beamer}
%**********************************************************************************************************************************************
\usepackage{multicol, stmaryrd, amsfonts, graphicx, times, epsfig, amsmath, mathtools, subfigure, balance, array, siunitx, pgfgantt, xspace}
\usepackage{multirow, epsf, amsmath, amssymb, indentfirst, verbatim, keyval, url, textcomp, enumerate, calc, makecell, subfigure, xcolor, listings}
%**********************************************************************************************************************************************
\usepackage[utf8]{inputenc}
\usepackage[export]{adjustbox}
\usepackage[absolute,overlay]{textpos}
\usepackage{hyperref}
\usepackage[normalem]{ulem}
\usepackage{textpos}
\usepackage{breakurl}
%**********************************************************************************************************************************************
\useunder{\uline}{\ulined}{}%
\DeclareUrlCommand{\bulurl}{\def\UrlFont{\ttfamily\color{blue}\ulined}}
\usefonttheme{serif}
\usetheme{Pittsburgh}
\usecolortheme{beaver}
\graphicspath{ {figs/} }
\setbeamertemplate{itemize items}[ball]
\hypersetup{
   colorlinks   = true,                               %Colours links instead of ugly boxes
   urlcolor     = blue,                               %Colour for external hyper links
   linkcolor    = blue,                               %Colour of internal links
   citecolor    = red,                                %Colour of citations
   setpagesize  = false,
   linktocpage  = true,
}
%**********************************************************************************************************************************************
%\usepackage{stmaryrd, amsfonts, graphicx, times, epsfig, amsmath, mathtools,subfigure, balance, array,siunitx}
%\usefonttheme{structuresmallcapsserif}
%\usepackage{bookman}
%\usetheme{Madrid}
%\usetheme{Szeged}
%**********************************************************************************************************************************************
 \AtBeginSection[]
{
  \begin{frame}
    \frametitle{Table of Contents}
    \tableofcontents[currentsection]
  \end{frame}
}
%**********************************************************************************************************************************************
%Information to be included in the title page:
\title[] %optional
{Graph Neural Network \\Lecture 2}

%**********************************************************************************************************************************************
%\author[Arthur, Doe] % (optional, for multiple authors)
%{A.~B.~Arthur\inst{1} \and J.~Doe\inst{2}}
%
%\institute[VFU] % (optional)
%{
%  \inst{1}%
%  Faculty of Physics\\
%  Very Famous University
%  \and
%  \inst{2}%
%  Faculty of Chemistry\\
%  Very Famous University
%}
%**********************************************************************************************************************************************
\date[] % (optional)

%\logo{\includegraphics[width=1in, height=0.3in]{GW_logo.eps}}
\titlegraphic{\includegraphics[width=2in, height=0.8in]{GW_logo.eps}}
%>>>>>>>>>>>>>>>>>>>>>>>>>>>>>>>>>>>>>>>>>>>>>>>>>>>>>>>>>>>>>>>>>>>>>>>>>>>>>>>>>>>>>>>>>>>>>>>>>>>>>>>>>>>>>>>>>>>>>>>>>>>>>>>>>>>>>>>>>>>>>>>
%>>>>>>>>>>>>>>>>>>>>>>>>>>>>>>>>>>>>>>>>>>>>>>>>>>>>>>>>>>>>>>>>>>>>>>>>>>>>>>>>>>>>>>>>>>>>>>>>>>>>>>>>>>>>>>>>>>>>>>>>>>>>>>>>>>>>>>>>>>>>>>>
\setbeamertemplate{bibliography item}[text]

\usepackage{setspace}
\definecolor{Code}{rgb}{0,0,0}
\definecolor{Decorators}{rgb}{0.5,0.5,0.5}
\definecolor{Numbers}{rgb}{0.5,0,0}
\definecolor{MatchingBrackets}{rgb}{0.25,0.5,0.5}
\definecolor{Keywords}{rgb}{0,0,1}
\definecolor{self}{rgb}{0,0,0}
\definecolor{Strings}{rgb}{0,0.63,0}
\definecolor{Comments}{rgb}{0,0.63,1}
\definecolor{Backquotes}{rgb}{0,0,0}
\definecolor{Classname}{rgb}{0,0,0}
\definecolor{FunctionName}{rgb}{0,0,0}
\definecolor{Operators}{rgb}{0,0,0}
\definecolor{Background}{rgb}{0.98,0.98,0.98}
\lstdefinelanguage{Python}{
numbers=left,
numberstyle=\footnotesize,
numbersep=1em,
xleftmargin=1em,
framextopmargin=2em,
framexbottommargin=2em,
showspaces=false,
showtabs=false,
showstringspaces=false,
frame=l,
tabsize=4,
% Basic
basicstyle=\ttfamily\small\setstretch{1},
backgroundcolor=\color{Background},
% Comments
commentstyle=\color{Comments}\slshape,
% Strings
stringstyle=\color{Strings},
morecomment=[s][\color{Strings}]{"""}{"""},
morecomment=[s][\color{Strings}]{'''}{'''},
% keywords
morekeywords={import,from,class,def,for,while,if,is,in,elif,else,not,and,or,print,break,continue,return,True,False,None,access,as,,del,except,exec,finally,global,import,lambda,pass,print,raise,try,assert},
keywordstyle={\color{Keywords}\bfseries},
% additional keywords
morekeywords={[2]@invariant,pylab,numpy,np,scipy},
keywordstyle={[2]\color{Decorators}\slshape},
emph={self},
emphstyle={\color{self}\slshape},
%
}
%-----------------------------------------------------------------------------------------------------------------------------------------------------------------------------
%-----------------------------------------------------------------------------------------------------------------------------------------------------------------------------
%-----------------------------------------------------------------------------------------------------------------------------------------------------------------------------
\begin{document}
\lstset{language=Python}
\frame{\titlepage}

%-----------------------------------------------------------------------------------------------------------------------------------------------------------------------------


\section{Introduction to PyG}
%------------------------------------------------
\begin{frame}{What is PyTorch Geometric?}
    \begin{itemize}
        \item \textbf{PyG} (\href{https://github.com/pyg-team/pytorch_geometric}{PyTorch Geometric}) is a library for deep learning on graphs built on PyTorch.
        \item Provides tools for graph creation, transformations, and GNN layers.
        \item Efficient handling of large-scale graphs.
         \item Installation:
         	\begin{itemize}
			\item \texttt{pip install torch\_geometric}
			\item  \texttt{conda install pyg \-c pyg}
		\end{itemize}
    \end{itemize}
\end{frame}

%------------------------------------------------
\begin{frame}{Key Features}
    \begin{itemize}
        \item \textbf{Data Handling}: Easy-to-use data structures for graphs.
        \item \textbf{Predefined Layers}: GCN, GAT, GraphSAGE, etc.
        \item \textbf{Datasets}: Built-in datasets like Cora, Citeseer, and more.
        \item \textbf{Extensibility}: Custom layers and datasets.
    \end{itemize}
\end{frame}


%-----------------------------------------------------------------------------------------------------------------------------------------------------------------------------
\section{Basic Code Examples}
%------------------------------------------------
\begin{frame}
\frametitle{Creating a simple graph}
\lstinputlisting[firstline=1, lastline=12]{Python_Files/Lecture2_Code.py}
\end{frame}

%-----------------------------------------------------------------------------------------------------------------------------------------------------------------------------
\begin{frame}
\frametitle{Accessing Graph Properties}
\lstinputlisting[firstline=15, lastline=18]{Python_Files/Lecture2_Code.py}
\end{frame}

%-----------------------------------------------------------------------------------------------------------------------------------------------------------------------------
\begin{frame}
\frametitle{Adding Edge Weights}
\lstinputlisting[firstline=32, lastline=35]{Python_Files/Lecture2_Code.py}
\end{frame}

%-----------------------------------------------------------------------------------------------------------------------------------------------------------------------------
\begin{frame}
\frametitle{Convert to Undirected Graph}
\lstinputlisting[firstline=38, lastline=40]{Python_Files/Lecture2_Code.py}
\end{frame}

%-----------------------------------------------------------------------------------------------------------------------------------------------------------------------------
\begin{frame}
\frametitle{Using Built-in Datasets}
\lstinputlisting[firstline=21, lastline=29]{Python_Files/Lecture2_Code.py}
\end{frame}


%---------------------------------------------
\section{Advanced Features}
%---------------------------------------------

\begin{frame}{Batching Graphs}
    \begin{itemize}
        \item PyG supports batching multiple graphs into a single graph object.
        \item Useful for training on datasets with multiple small graphs.
    \end{itemize}
\lstinputlisting[firstline=43, lastline=49]{Python_Files/Lecture2_Code.py}
\end{frame}

%---------------------------------------------
\begin{frame}{Graph Transformations}
    \begin{itemize}
        \item PyG provides utilities for graph transformations (e.g., normalization, augmentation).
    \end{itemize}
\lstinputlisting[firstline=52, lastline=57]{Python_Files/Lecture2_Code.py}
\end{frame}

%---------------------------------------------
\begin{frame}
\frametitle{Using a GCN Layer}
\lstinputlisting[firstline=60, lastline=73]{Python_Files/Lecture2_Code.py}
\end{frame}

%---------------------------------------------
\begin{frame}
\frametitle{Defining a Custom GNN Layer}
\lstinputlisting[firstline=76, lastline=83]{Python_Files/Lecture2_Code.py}
\end{frame}

%---------------------------------------------
\begin{frame}
\frametitle{Training a GNN Model}
\lstinputlisting[firstline=86, lastline=95]{Python_Files/Lecture2_Code.py}
\end{frame}

\section{Conclusion}
%---------------------------------------------
\begin{frame}{Conclusion}
    \begin{itemize}
        \item PyTorch Geometric simplifies working with GNNs.
        \item Supports various graph neural network architectures.
        \item Efficient for large-scale graph data processing.
    \end{itemize}
\end{frame}
\end{document}

